\input{Configuraciones/paquetes}

%--------------------------

\begin{document}
\input{Configuraciones/nombres}
%--------------------------
\begin{problema}
	¿Cuáles crees que fueron las acciones correctas que tomaron para lograr el éxito?
\end{problema}
\begin{sol}
	Las acciones correctas probablemente fueron las siguientes: \begin{enumerate}
		\item Lanzar un concepto nuevo de alquiler de vestidos con elementos ya preexistentes de otras empresas que ya estaban en el sector. 
		\item Buscar asociarse con diseñadores de moda famosos y explicarles cual era el verdadero concepto de la empresa que se iba a lanzar. 
		\item Evaluar en pequeños grupos el comportamiento de los consumidores y en base a eso definir nuevas estrategias. 
		\item Crear un equipo homogéneo y dispuestos a dar un buen desempeño. 
		\item Centrarse en un único segmento y no expandirse demasiado. 
	\end{enumerate}
\end{sol}
\begin{problema}
	¿Qué estrategia lograron usar para convencer a los diseñadores?
\end{problema}
\begin{sol}
	En primer lugar, desistieron de alquilar los vestidos que fueran el último hito; pero sí alquilar vestidos que fueran bastante recientes. La idea fue que en lugar de pedirles directamente los vestidos a los diseñadores, las cofundadoras fueron pidiendo consejos. 
\end{sol}
\begin{problema}
	¿Cuál fue su estrategia de mercado para obtener clientes y crecer?
\end{problema}
\begin{sol}
	Los mismos vestidos se promocionaban así mismos cuando las personas lo llevaban puestos y así sucesivamente se iban expandiendo. Además, la empresa contaba con varios servicios que permitían que en caso un vestido no fuera de la talla correcta, se podía elegir varios con un costo mínimo extra. 
\end{sol}
\begin{problema}
	¿Por qué lograron atraer capital de riesgo?
\end{problema}
\begin{sol}
	La idea era innovadora y los resultados eran bastante satisfactorios. 
\end{sol}
\begin{problema}
	¿Cuál es la decisión que deben tomar a inicios del 2010 y qué decisión tomarían ustedes?
\end{problema}
 \begin{sol}
 	Deben decidir si expandir la empresa a otro tipo de productos y generar más ingresos; a costo de sacrificar un crecimiento más lento pero seguro. En mi caso personal, preferiría esperar hasta que todo esté en orden y pueda llevar una expansión mucho más grande. 
 \end{sol}


%---------------------------
%\bibliographystyle{apa}
%\bibliography{referencias.bib}

\end{document}