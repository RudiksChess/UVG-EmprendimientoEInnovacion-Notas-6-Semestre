\input{Configuraciones/paquetes}

%--------------------------

\begin{document}
\input{Configuraciones/nombres}
%--------------------------

\section{Negocio}

El negocio elegido: comida rápida. 

\section{Segmento de cliente }

\begin{enumerate}
	\item Sexo: indistinto.
	\item Edad: 18-50 años (clase trabajadora).
	\item Nivel socioeconómico: clase media-trabajadora. 
	\item Hábitos: \begin{enumerate}
		\item Horarios apretados. 
		\item Carece de tiempo para comer. 
		\item Suele comer comida chatarra. 
	\end{enumerate}
\end{enumerate}

\section{Cliente específico}

\begin{enumerate}
	\item Es una persona que se viste formal, debido a su trabajo de oficinista. 
	\item Es una persona que tiene hábitos alimenticios cuestionables, se suele alimentar de comida chatarra. 
	\item Es una persona que no suele hacer ejercicio. 
	\item Dispone de poco tiempo para alimentarse. 
\end{enumerate}

\includepdf[landscape= true, pages=-]{apendices/mapaempatia.pdf}

%---------------------------
%\bibliographystyle{apa}
%\bibliography{referencias.bib}

\end{document}