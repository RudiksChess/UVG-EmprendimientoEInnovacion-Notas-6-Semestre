\documentclass[a4paper,12pt]{article}
\usepackage[top = 2.5cm, bottom = 2.5cm, left = 2.5cm, right = 2.5cm]{geometry}
\usepackage[T1]{fontenc}
\usepackage[utf8]{inputenc}
\usepackage{multirow} 
\usepackage{booktabs} 
\usepackage{graphicx}
\usepackage[spanish]{babel}
\usepackage{setspace}
\setlength{\parindent}{0in}
\usepackage{float}
\usepackage{fancyhdr}
\usepackage{amsmath}
\usepackage{amssymb}
\usepackage{amsthm}
\usepackage[numbers]{natbib}
\newcommand\Mycite[1]{%
	\citeauthor{#1}~[\citeyear{#1}]}
\usepackage{graphicx}
\usepackage{subcaption}
\usepackage{booktabs}
\usepackage{etoolbox}
\usepackage{minibox}
\usepackage{hyperref}
\usepackage{xcolor}
\usepackage{pdfpages}
\usepackage[skins]{tcolorbox}
%---------------------------

\newtcolorbox{cajita}[1][]{
	 #1
}

\newenvironment{sol}
{\renewcommand\qedsymbol{$\square$}\begin{proof}[\textbf{Solución.}]}
	{\end{proof}}

\newenvironment{dem}
{\renewcommand\qedsymbol{$\blacksquare$}\begin{proof}[\textbf{Demostración.}]}
	{\end{proof}}

\newtheorem{problema}{Problema}
\newtheorem{definicion}{Definición}
\newtheorem{ejemplo}{Ejemplo}
\newtheorem{teorema}{Teorema}
\newtheorem{corolario}{Corolario}[teorema]
\newtheorem{lema}[teorema]{Lema}
\newtheorem{prop}{Proposición}
\newtheorem*{nota}{\textbf{NOTA}}
\renewcommand\qedsymbol{$\blacksquare$}
\usepackage{svg}
\usepackage{tikz}
\usepackage[framemethod=default]{mdframed}
\global\mdfdefinestyle{exampledefault}{%
linecolor=lightgray,linewidth=1pt,%
leftmargin=1cm,rightmargin=1cm,
}




\newenvironment{noter}[1]{%
\mdfsetup{%
frametitle={\tikz\node[fill=white,rectangle,inner sep=0pt,outer sep=0pt]{#1};},
frametitleaboveskip=-0.5\ht\strutbox,
frametitlealignment=\raggedright
}%
\begin{mdframed}[style=exampledefault]
}{\end{mdframed}}
\newcommand{\linea}{\noindent\rule{\textwidth}{3pt}}
\newcommand{\linita}{\noindent\rule{\textwidth}{1pt}}

\AtBeginEnvironment{align}{\setcounter{equation}{0}}
\pagestyle{fancy}

\fancyhf{}









%----------------------------------------------------------
\lhead{\footnotesize Emprendimiento e Innovación}
\rhead{\footnotesize  Rudik Roberto Rompich}
\cfoot{\footnotesize \thepage}


%--------------------------

\begin{document}
 \thispagestyle{empty} 
    \begin{tabular}{p{15.5cm}}
    \begin{tabbing}
    \textbf{Universidad del Valle de Guatemala} \\
    Departamento de Matemática\\
    Licenciatura en Matemática Aplicada\\\\
   \textbf{Estudiante:} Rudik Roberto Rompich\\
   \textbf{Correo:}  \href{mailto:rom19857@uvg.edu.gt}{rom19857@uvg.edu.gt}\\
   \textbf{Carné:} 19857
    \end{tabbing}
    \begin{center}
        CU1778 - Emprendimiento e Innovación - Catedrático: Julio Martínez\\
        \today
    \end{center}\\
    \hline
    \\
    \end{tabular} 
    \vspace*{0.3cm} 
    \begin{center} 
    {\Large \bf  Caso 2
} 
        \vspace{2mm}
    \end{center}
    \vspace{0.4cm}
%--------------------------


\begin{problema}
	¿Por qué Chris pensaba que debían sacar una cerveza light al mercado?
\end{problema}

\begin{sol}
	Por lo siguientes motivos: 
	\begin{enumerate}
		\item Por primera vez en la historia de la compañía de cervezas de su padre se estaban experimentando pérdidas.
		\item  Según las estadísticas, el sector con mayor interés y crecimiento eran los jóvenes; los cuales se inclinaban a cervezas del tipo \textit{light}. 
		\item Según los datos recopilados por un grupo focal, la mayoría de gente percibía a la cerveza actual como un producto para trabajadores o gente vieja. 
		\item La percepción de Chris era a futuro, ya que dentro de poco él dirigiría la compañía. 
	\end{enumerate}
\end{sol}

%--------------------

\begin{problema}
	¿Por qué Oscar y el resto de ejecutivos no querían producir una cerveza light? ¿Cómo les hubieras mostrado tu perspectiva?
\end{problema}
\begin{sol}
	Existían diversos motivos entre los cuales resaltan: 
	\begin{enumerate}
		\item No querían que el producto principal quedará relegado por otro. 
		\item Existía mucha competencia de compañías mucho más grandes y robustas. 
		\item El lanzamiento de una nueva cerveza implicaría una campaña publicitaria que podría ser muy cuantiosa. 
		\item Según los datos proporcionados, la mayoría de clientes jóvenes se inclinaban a comprar marcas \textit{mainstream}, en lugar de marcas tradicionales. 
		\item Se pensaba que la introducción de un nuevo producto implicaría un gasto enorme en la producción, que representaría incluso menos ganancias a largo plazo. 
	\end{enumerate}

Por otra parte, en mi opinión personal, mi perspectiva la hubiera planteado desde una perspectiva puramente estadística; mostrando análisis de riesgos, beneficios y posibles pérdidas. A partir de eso, formular una estrategia mucho más certera y objetiva basada en la evidencia. 
\end{sol}

%--------------------

\begin{problema}
	¿Cuál es el perfil del cliente actual de Mountain Man?
\end{problema}

\begin{sol}
	\begin{enumerate}
		\item Clientes de «la vieja escuela», consideran a la cerveza como una tradición en las regiones donde opera la compañía.
		\item Personas maduras que ven a la cerveza como algo para sentirse orgullosos. 
		\item Hombres, principalmente.  
	\end{enumerate}
\end{sol}


%--------------------

\begin{problema}
	¿Qué hubieras hecho en el caso de Chris?
\end{problema}
\begin{sol}
	Supongo que observaría a mayor profundidad la tendencia del mercado, las ganancias actuales-futuro, la necesidad de generar campañas publicitarias, quizás lanzar el producto a pequeña escala y ver como funciona. Como nos cuentan el caso, tengo la impresión que una cerveza \textit{light} es la tendencia que predominará en el futuro cercano; sin embargo, el problema es generar un nicho en el mercado y competir con las grandes compañías. Así que probablemente, evaluaría primero todos los escenarios posibles y observaría a la competencia que tenga similares condiciones a la compañía de cervezas de Chris. 

\end{sol}
%---------------------------
%\bibliographystyle{apa}
%\bibliography{referencias.bib}

\end{document}