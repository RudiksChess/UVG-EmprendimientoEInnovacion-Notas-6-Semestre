\input{Configuraciones/paquetes}

%--------------------------

\begin{document}
\input{Configuraciones/nombres}
%--------------------------

\textbf{Instrucciones:} 
\begin{enumerate}
	\item No es necesario utilizar el diagrama.
	\item Incluir TAM, SAM y SOM para sus proyectos individuales.
	\item Incluir las referencias y/o lógica que utilizaron para llegar a esta conclusión.
	\item Presentarlo en formato texto, no es necesario presentarlo en clase.
\end{enumerate}

\section{Esquema}

\subsection{TAM - Total Market Available}
\begin{enumerate}
	\item Empresa basada en Guatemala como país. 
	\item Los sectores productivos en los que los servicios de mi compañía impactar, basados en sus porcentajes de representación en la economía guatemalteca basado en la encuesta nacional de empleo e ingresos\cite{ine_guatemala_encuesta_2019} de 2019: 
	\begin{enumerate}
		\item Actividades de administración pública - 9.2\% de la actividad económica. 
		\item Actividades profesionales y científicas - 3.2\% de la actividad económica. 
		\item Actividades financieras y de seguros - 1.2\% de la actividad económica. 
		\item Información y comunicación - 0.6\% de la actividad económica . 
		\item Actividades inmobiliarias - 0.3\% de la actividad económica. 
	\end{enumerate}


\end{enumerate}
De acuerdo a lo anterior, los sectores mencionados conforman alrededor del 14.5 \% de la actividad económica de Guatemala. Las empresas que conforman estos sectores serían hasta cierto punto un enfoque global al que intentaría llegar mi empresa si no existiera ninguna limitante. 

\subsection{SAM - Serviceable Available Market}

En caso que existieran limitantes económicos, financieros, falta de recursos esenciales de las empresas de permitirse contratar los servicios de mi compañía, entonces los sectores principales serían: 
\begin{enumerate}
	\item Actividades financieras y de seguros.
	\item Información y comunicación.
	\item Actividades inmobiliarias.
\end{enumerate}

Estos sectores son casi una garantía que tendrían los recursos monetarios y un sistema de almacenamiento de datos (la empresa está basada en datos) ya previamente establecida. Por lo tanto, serían los clientes potenciales que podrían contratar y permitirse una consultora para manejar sus datos. 


\subsection{SOM -  Serviceable Obtainable Market}

Las empresas que realmente se podrían alcanzar son las que se catalogan como medianas y grandes empresas; esto debido a que cuentan con los siguientes factores: 

\begin{enumerate}
	\item Tienen bases de datos robustas. 
	\item Tienen la capacidad económica de contratar una consultora. 
	\item Son empresas que ya están establecidas y que buscan optimizar sus servicios. 
\end{enumerate}

Según CentralAmericaData\cite{central_america_data_clasificacion_2015} las empresas mediadas se catalogan como: 

\begin{enumerate}
	\item Tienen entre 81 y 200 trabajadores. 
	\item  Ventas anuales de entre 3,701 a 15,420 salarios mínimos no agrícolas. (i.e Entre Q 9,364,788.30 y Q 39,017,843.00). 
\end{enumerate}

Mientras que las grandes empresas, son las que superan las cifras de las medianas empresas. 



%---------------------------
\bibliographystyle{plain}
\bibliography{referencias.bib}

\end{document}